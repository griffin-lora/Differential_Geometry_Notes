\documentclass[notes.tex]{subfiles}

\begin{document}

\setcounter{section}{2}
\section{Curves}

\subsection{Definition of a Curve}
\begin{definition}[$1$-Manifold]
    A \textit{$1$-manifold} or \textit{curve} is a set $\Omega \subseteq \real^n$ which locally looks like a line near any of its points, called its \textit{tangent line} at $P \in \Omega$.
\end{definition}
\begin{definition}[Orientable $1$-Manifold]
    An \textit{orientable $1$-manifold} is a \textit{$1$-manifold} $\Omega$ such that there is a specific ``orientation" for $\Omega$ where $\Omega'$ has the opposite orientation and $(\Omega')' = \Omega$.
\end{definition}
\begin{enumerate}[label = \arabic*)]
    \item We can describe manifolds in different but equivalent ways. Notice that $(x, y)$ st $x^2 + y^2 = 1$ and $(\cos t, \sin t)$, $t \in [0, 2\pi]$ produce precisely the same set.
    \item And working with forms in terms of parameterizations is much easier.
\end{enumerate}

\begin{definition}[Parameterized Curve]
    $\gamma$ is a \textit{parameterization} of a curve $C$ iff $\gamma\colon [a, b] \to \real^n$ is $C^1$ smooth and $\range \gamma = C$.
\end{definition}
Note that a parameterization induces a specific orientation on a curve. We will cover this later.

\begin{definition}[Regularity]
    $\gamma$ is \textit{regular} iff $\gamma'(t) \neq 0$ for all $t \in \dom \gamma$.
\end{definition}

\subsection{Integrating 1-Forms over Curves}
If we had to construct $P_i$ and $\vec{v_i}$ for $i = 1$ to $k$ for every integration it would be very cumbersome. However, we can take advantage of some properties of parameterization of curves. \\
For the following derivation, let $\gamma\colon[a, b] \to \real^n$ be a parameterization. Now we take a partition of $[a, b]$ as $[t_i]_{i = 0}^k$ such that $a = t_0$, $b = t_k$, and $[t_{i - 1} < t_i]_{i = 1}^k$. We can now send this through the map $\gamma$ to get a partition of the curve $C$. Taylor's theorem (the one used for power series) states that for $i = 0$ to $k$ that
\[
    \gamma(t_i) - \gamma(t_{i - 1}) = \gamma'(t_{i - 1})(t_i - t_{i -1}) + E|t_i - t_{i - 1}|
\]
Then by black magic we let $L_i$ be the line segment from $\gamma(t_{i - 1})$ to $\gamma(t_{i - 1}) + \gamma'(t_{i - 1})(t_i - t_{i - 1})$ and this is a good approximation? Therefore we let $P_i = \gamma(t_{i - 1})$ and $\vec{v_i} = \gamma'(t_{i - 1})(t_i - t_{i - 1})$. Now we consider the Riemann sum for some arbitrary $1$-form $\omega$ on $C$ as
\begin{align*}
    \sum_{i = 1}^k \omega_{P_i}(\vec{v_i})
    &= \sum_{i = 1}^k \omega_{\gamma(t_{i - 1})}(\gamma'(t_{i - 1})(t_i - t_{i - 1})) \\
    &= \sum_{i = 1}^k \sum_{j = 1}^n f_j(\gamma(t_{i - 1}))(dx_i)_{\gamma(t_{i - 1})}(\gamma'(t_{i - 1})(t_i - t_{i - 1})) \\
    % what?
    &= \sum_{i = 1}^k \sum_{j = 1}^n f_j(\gamma(t_{i - 1}))\gamma'(t_{i - 1})(t_i - t_{i - 1})
\end{align*}
However notice that
\[
    \lim_{k\to\infty} \sum_{i = 1}^k \sum_{j = 1}^n f_j(\gamma(t_{i - 1}))\gamma'(t_{i - 1})(t_i - t_{i - 1}) = \int_a^b \sum_{j = 1}^n f_j(\gamma(t))\gamma'(t)dt
\]
Therefore
\[
    \lim_{k\to\infty} \sum_{i = 1}^k \omega_{P_i}(\vec{v_i}) = \int_a^b \sum_{j = 1}^n f_j(\gamma(t))\gamma'(t)dt = \int_a^b \omega_{\gamma(t)}(\gamma'(t))dt
\]

Notice that $\gamma(t) = (\cos t, \sin t)$, $t \in [0, \pi]$ and $\phi(t) = (-t, \sqrt{1 - t^2})$, $t \in [-1, 1]$ both paramaterize the upper-half plane cut of the unit circle $C = \{ (x, y) : x^2 + y^2 = 1, y \geq 0 \}$.

We now need to show that the choice of paramaterization matters only up to orientation (i.e a paramaterization that induces an opposite orientation will negate the integral).
\begin{theorem}[Uniqueness of Integration up to Orientation] \label{integral_uniqueness_up_to_orientation_one_form}
    If we have paramaterizations $\gamma\colon [a, b] \to \real^n$ and $\phi\colon [a, b] \to \real^n$ for $C$ where there exists a map $\psi\colon[a, b] \to [c, d]$ such that
    \begin{enumerate}[label = \arabic*)]
        \item $\psi$ is invertible and $\psi, \psi^{-1}$ are of smoothness $C^1$;
        \item $\gamma(t) = \phi(\psi(t))$ for all $t \in [a, b]$; and
        \item $\psi'(t) > 0$ for all $t \in [a, b]$,
    \end{enumerate}
    then for all $1$-forms $\omega$ over $C$
    \[
        \int_a^b \omega_{\gamma(t)}(\gamma'(t))dt = \int_c^d \omega_{\phi(t)}(\phi'(t))dt
    \]
\end{theorem}
\begin{proof}
    \begin{align*}
        \int_a^b \omega_{\gamma(t)}(\gamma'(t))dt
        &= \int_a^b \omega_{\phi(\psi(t))}(\phi'(\psi(t))\psi'(t))dt \\
        &= \int_a^b \omega_{\phi(\psi(t))}(\phi'(\psi(t)))\psi'(t)dt \\
    \end{align*}
    Now we can substitute $u = \psi(t)$ so $du = \psi'(t)dt$. Now consider the new bounds $\alpha = \psi(a)$ to $\beta = \psi(b)$. Since $\psi'(t) > 0$ for $t \in [a, b]$ then $\psi$ has no local extrema by Fermat's theorem. Therefore, by the extreme value theorem, $\psi$ has extrema of $\psi(a)$ and $\psi(b)$. Also notice that since $\psi$ is monotonically increasing, $\psi(a) < \psi(b)$, therefore $\psi(a)$ is the absolute maximum and $\psi(b)$ is the absolute minimum. Since $\psi$ is surjective, then $\psi(t)$ it attains all values on $[c, d]$, therefore it must have a minimum of $c$ and maximum of $d$. Therefore, $\psi(a) = c$ and $\psi(b) = d$ so
    \[
        \int_a^b \omega_{\phi(\psi(t))}(\phi'(\psi(t)))\psi'(t)dt
        = \int_c^d \omega_{\phi(u)}(\phi'(u))du
    \]
    So by redefining $u = t$
    \[
        \int_a^b \omega_{\gamma(t)}(\gamma'(t))dt = \int_c^d \omega_{\phi(t)}(\phi'(t))dt
    \]
\end{proof}

\begin{definition}[Integral of a $1$-Form over a Curve]
    Given a curve $C \in \real^n$ from $A$ to $B$ and a $1$-form $\omega$ defined on $C$ then the \textit{integral} of $\omega$ over $C$ is
    \[
        \int_C \omega = \lim_{k\to\infty} \sum_{i = 1}^k \omega_{P_i}(\vec{v_i})
    \]
    where $[P_i]_{i = 0}^k \in C$ such that $P_0 = A$ and $P_k = B$ and $[\vec{v_i} = \overrightarrow{P_{i - 1}P_i}]_{i = 1}^k$.
\end{definition}

\begin{theorem}
    Given a parameterization $\gamma\colon [a, b] C$ and a $1$-form $\omega$ defined on $C$ then
    \[
        \int_C \omega = \int_a^b \omega_{\gamma(t)}(\gamma'(t))dt
    \]
\end{theorem}
\begin{proof}
    Consider the partition of $[a, b]$ as $[t_i]_{i = 0}^k$ such that $a = t_0$, $b = t_k$, and $[t_{i - 1} < t_i]_{i = 1}^k$. Now consider the partition it induces on the curve through $\gamma$ so $P_i = \gamma(t_{i - 1})$ and $\vec{v_i} = \gamma(t_i) - \gamma(t_{i - 1})$ for $i = 1$ to $k$. Therefore
    \begin{align*}
        \int_C \omega
        &= \lim_{k\to\infty} \sum_{i = 1}^k \omega_{P_i}(\vec{v_i})
        = \lim_{k\to\infty} \sum_{i = 1}^k \omega_{P_i}(\gamma(t_i) - \gamma(t_{i - 1})) \\
        &= \lim_{k\to\infty} \sum_{i = 1}^k \omega_{P_i}(\gamma'(t_{i - 1})(t_i - t_{i -1}) + E|t_i - t_{i - 1}|) \quad \text{By Taylor's Theorem} \\
        &= \lim_{k\to\infty} \sum_{i = 1}^k \omega_{P_i}(\gamma'(t_{i - 1})(t_i - t_{i -1})) \quad \text{Since $t_i - t_{i - 1} \to 0$} \\
        &= \lim_{k\to\infty} \sum_{i = 1}^k \omega_{\gamma(t_{i - 1})}(\gamma'(t_{i - 1}))(t_i - t_{i -1}) \\
        &= \int_a^b \omega_{\gamma(t)}(\gamma'(t))dt \\
    \end{align*}
    By \cref{integral_uniqueness_up_to_orientation_one_form}, this integral is unique regardless of choice of paramaterization up to orientation.
\end{proof}

\begin{exercise}
    Let $\omega = x^2dx + dy$ and let $C$ with parameterization $\gamma(t) = (t, t^2)$ for $t \in [0, 1]$. Find $\int_C \omega$.
\end{exercise}
\begin{solution}
    \begin{align*}
        \int_C \omega
        &= \int_0^1 \omega_{\gamma(t)}(\gamma'(t))dt
        = \int_0^1 (t^2dx + dy)_{(t, t^2)}(1, 2t)dt \\
        &= \int_0^1 (t^2(dx)_{(t, t^2)}(1,2 t) + (dy)_{(t, t^2)}(1, 2t))dt \\
        &= \int_0^1 (t^2 + 2t)dt
        = \frac{4}{3}
    \end{align*}
\end{solution}

\begin{theorem}[Fundamental Theorem of Line Integrals]
    If $C$ is a curve in $\real^n$, $f\colon C \to \real$, then for any paramaterization $\gamma\colon [a, b] \to \real^n$ of $C$ that
    \[
        \int_C (df)(P) = f(\gamma(b)) - f(\gamma(a))
    \]
\end{theorem}
\begin{proof}
    Let $\gamma = \sum_{i = 1}^n \gamma_i\vec{v_i}$. Then by the chain rule
    \[
        \frac{d}{dt} f(\gamma(t)) = \sum_{i = 1}^n \frac{\del f}{\del x_i}(\gamma(t))\gamma'_i(t)
    \]
    Therefore
    \begin{align*}
        \int_C (df)(P)
        &= \int_C \sum_{i = 1}^n \frac{\del f}{\del x_i}(P)dx_i \\
        &= \int_a^b \sum_{i = 1}^n \frac{\del f}{\del x_i}(\gamma(t))(dx_i)_{\gamma(t)}(\gamma'(t))dt \\
        &= \int_a^b \sum_{i = 1}^n \frac{\del f}{\del x_i}(\gamma(t)) \gamma'_i(t) dt \\
        &= \int_a^b \frac{d}{dt}f(\gamma(t))dt \\
        &= f(\gamma(b)) - f(\gamma(a))
    \end{align*}
\end{proof}

\end{document}