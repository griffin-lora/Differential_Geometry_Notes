\documentclass[notes]{subfiles}
\externaldocument{determinants}

\begin{document}

\setcounter{section}{6}
\section{Differential k-forms}
\subsection{Definition of k-forms}

\begin{definition}[Differential $k$-Form]
    $\alpha$ is a \textit{$k$-form} \textit{on} $\real^n$ iff $\alpha$ is a map from every $k$-parallelotope in $\real^n$ to a scalar in $\real$ such that $\alpha$ is
    \begin{enumerate}[label = (\arabic*)]
        \item scaling with respect to the scale of any spanning vector
        \[
            \alpha_P(\ldots, t\vec{v}, \ldots) = t\alpha_P(\ldots, \vec{v}, \ldots)
        \]
        \item additive for each spanning vector
        \[
            \alpha_P(\ldots, \vec{u} + \vec{v}, \ldots) = \alpha_P(\ldots, \vec{u}, \ldots) + \alpha_P(\ldots, \vec{v}, \ldots)
        \]
        \item alternating with respect to swapping any two spanning vectors
        \[
            \alpha_P(\ldots, \vec{u}, \ldots, \vec{v}, \ldots) = -\alpha_P(\ldots, \vec{v}, \ldots, \vec{u}, \ldots)
        \]
        \item $C^1$ smooth with respect to $P$
    \end{enumerate}
\end{definition}

\begin{example}
    Consider the $2$-form $dx \wedge dy$ on $\real^3$ such that
    \[
        (dx \wedge dy)_P(\vec{v}, \vec{w}) =
        \begin{vmatrix}
            a_1 & b_1 \\
            a_2 & b_2
        \end{vmatrix}
    \]
    By the axioms of a determinant, $dx \wedge dy$ (and all other basis exterior product forms) follow axioms 1--3 of a $k$-form. As long as the function in front of the determinant is smooth with respect to $P$ then we get a $k$-form. In this case it is the constant function so this is a $2$-form.

    Also notice that $(dx \wedge dy)_P(\vec{u}, \vec{v})$ will give us the oriented area of the parallelogram spanned by $\vec{u}$ and $\vec{v}$ when projected down to the $xy$ plane.

    Similarly we can define $2$-forms
    \[
        (dx \wedge dz)_P(\vec{v}, \vec{w}) =
        \begin{vmatrix}
            a_1 & b_1 \\
            a_3 & b_3
        \end{vmatrix}
    \]
    and
    \[
        (dy \wedge dz)_P(\vec{v}, \vec{w}) =
        \begin{vmatrix}
            a_2 & b_2 \\
            a_3 & b_3
        \end{vmatrix}
    \]
    which will give us the oriented area of the parallelogram spanned by $\vec{u}$ and $\vec{v}$ when projected down to the $xz$ and $yz$ planes respectively.
\end{example}

\begin{example}
    Let $\vec{F}\colon \real^3 \to \real^3$ be a $C^1$ smooth vector field. We can think of $\vec{F}(P)$ as describing the velocity of a fluid at point $P$.

    Now consider a point $P$ and the parallelogram spanned by $\vec{u}$ and $\vec{v}$ at $P$. We can now choose a normal basis $\beta$ isomorphic to $\real^2$ such that $\vec{u} \in \lspan\beta$ and $\vec{v} \in \lspan\beta$. We can now orthonormally extend $\beta$ to a new basis $\beta' = \beta \cup \{ \vec{n} \}$ isomorphic to $\real^3$. We can now define the flux through $P$ as
    \[
        (\vec{F}(P)\cdot\vec{n})\det_\beta(\vec{u}, \vec{v})
    \]
    Notice that $\vec{F}(P)\cdot\vec{n}$ is the signed magnitude of $\vec{F}(P)$ that does not lie in $\lspan \beta$ the flux through $P$ will scale the area of the parallelogram formed by $\vec{u}$ and $\vec{v}$ by exactly the signed magnitude of $\vec{F}(P)$ orthogonal to the parallelogram. Therefore the flux is equal to the volume of the parallelopiped spanned by $\vec{u}$, $\vec{v}$, and $\vec{F}(P)$ or
    \[
        \det(v, w, \vec{F}(P))
    \]
    We now associate a flux form to $\vec{F}$ as the $2$-form $\omega$ defined by
    \begin{align*}
        \omega_P(\vec{u}, \vec{v})
        &= \det(\vec{u}, \vec{v}, \vec{F}(P))
        = \begin{vmatrix}
            a_1 & b_1 & F_1(P) \\
            a_2 & b_2 & F_2(P) \\
            a_3 & b_3 & F_3(P)
        \end{vmatrix} \\
        &= F_1(P)\begin{vmatrix}
            a_2 & b_2 \\
            a_3 & b_3
        \end{vmatrix}
        - F_2(P)\begin{vmatrix}
            a_1 & b_1 \\
            a_3 & b_3
        \end{vmatrix}
        + F_3(P)\begin{vmatrix}
            a_1 & b_1 \\
            a_2 & b_2
        \end{vmatrix}
    \end{align*}
    Therefore we can write $\omega$ as
    \[
        \omega = F_1(P)dy\wedge dz - F_2(P)dx\wedge dz + F_3(P)dx \wedge dy
    \]
\end{example}

\end{document}