\documentclass[notes.tex]{subfiles}

\begin{document}

\setcounter{section}{4}
\section{Determinants}

\subsection{Definition and Basic Properties}
\begin{definition}[Determinant]
    A \textit{determinant} of \textit{dimension} $n$ is a function
    \[
        D\colon \bigtimes_{i = 1}^n \real^n \to \real
    \]
    such that it is
    \begin{enumerate}[label = (\arabic*)]
        \item invariant under the scaling of any of its column vectors
        \[
            D\left(\begin{cases} 
                t\vec{v_i} & i = j \\
                \vec{v_i} & .
            \end{cases}
            \right)_{i = 1}^n = tD(\vec{v_i})_{i = 1}^n
        \]
        \item anticommutative with respect to swapping of column vectors
        \[
            D\left(\begin{cases}
                \vec{v_j} & i = j \\
                \vec{v_k} & i = k \\
                \vec{v_i} & .
            \end{cases}\right)_{i = 1}^n
            =
            -D\left(\begin{cases}
                \vec{v_k} & i = j \\
                \vec{v_j} & i = k \\
                \vec{v_i} & .
            \end{cases}\right)_{i = 1}^n
        \]
        \item invariant under adding linear combinations of column vectors together
        \[
            D\left(\begin{cases}
                \vec{v_i} + t\vec{v_j} & i = k \\
                \vec{v_i} & .
            \end{cases}\right)_{i = 1}^n = D(\vec{v_i})_{i = 1}^n
        \]
        \item identically one on basis column vectors
        \[
            D(\vec{e_i})_{i = 1}^n = 1
        \]
        where $\{\vec{e_i}\}_{i = 1}^n$ is a basis of $\real^n$
    \end{enumerate}
    
\end{definition}

\begin{lemma}
    The determinant is unique with respect to some given dimension $n$ and basis $\beta$.
\end{lemma}

Therefore we denote the determinant $D$ for the current context as $D = \det$.

\begin{theorem}
    \[
        \det\left( \begin{cases}
            \vec{u} + \vec{w} & i = j \\
            \vec{v_i} & .
        \end{cases} \right)
        =
        \det\left( \begin{cases}
            \vec{u} & i = j \\
            \vec{v_i} & .
        \end{cases} \right)
        +
        \det\left( \begin{cases}
            \vec{w} & i = j \\
            \vec{v_i} & .
        \end{cases} \right)
    \]
\end{theorem}

\subsection{Relationship to Volume}
First, we look at parallelograms in $\real^2$.

\begin{definition}[Parallelogram]
    If $\vec{v}, \vec{w} \in \real^2$ then
    \[
        P(\vec{v}, \vec{w}) = \{ s\vec{v} + t\vec{w} : 0 \leq s \leq 1, 0 \leq t \leq 1 \}
    \]
    is the \textit{parallelogram} that is \textit{spanned} by $\vec{v}$ and $\vec{w}$.
\end{definition}

If we change the order of $\vec{v}$ and $\vec{w}$ then we get the corresponding parallelogram with opposite orientation.

\begin{definition}[Oriented Area]
    $A\colon \real^2 \times \real^2 \to \real^2$ is the \textit{oriented area} function for parallelograms iff $A(\vec{v}, \vec{w})$ gives the area of $P(\vec{v}, \vec{w})$ if $P(\vec{v}, \vec{w})$ has positive orientation and $A(\vec{v}, \vec{w})$ gives the negative area of $P(\vec{v}, \vec{w})$ if $P(\vec{v}, \vec{w})$ has negative orientation.
\end{definition}

\begin{lemma}
    The oriented area function is equal to the determinant.
\end{lemma}
\begin{proof}
    For axiom 1, scaling any side of a parallelogram by $t$ will scale the area by $t$ so $A(t\vec{v}, \vec{w}) = A(\vec{v}, t\vec{w}) = tA(\vec{v}, \vec{w})$.

    For axiom 2, $A(\vec{v}, \vec{w}) = -A(\vec{w}, \vec{v})$ since swapping $\vec{v}$ and $\vec{w}$ switches the orientation.

    For axiom 3, $P(\vec{v} + t\vec{w}, \vec{w})$ and $P(\vec{v}, \vec{w} + t\vec{v})$ are sheared parallelograms, since area is invariant under shearing then $A(\vec{v} + t\vec{w}, \vec{w}) = A(\vec{v}, \vec{w} + t\vec{v}) = A(\vec{v},vec{w})$.

    For axiom 4, since the unit basis vectors all have length $1$, then the area must be $1$, therefore $A((1, 0), (0, 1)) = 1$.

    Therefore $A = \det$.
\end{proof}

\begin{theorem}
    The oriented volume of an $n$-parallelotope spanned by $[\vec{v_i}]_{i = 1}^n$ is $\det(\vec{v_i})_{i = 1}^n$.
\end{theorem}

\end{document}