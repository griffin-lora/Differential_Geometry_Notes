\documentclass[notes]{subfiles}

\begin{document}

\setcounter{section}{5}
\section{Top Dimensional Forms}
\subsection{Definition of Top Dimensional Forms}

\begin{definition}[Differential $n$-Form]
    $\omega$ is an \textit{$n$-form} or \textit{top form} on $\real^n$ iff %$\omega_P \in \mathcal{L}(\real^n, \real)$
    $\omega_P\colon \real^n \to \real$ such that
    \[
        \omega_P(\vec{v}_i)_{i = 1}^n = f(P)\det(\vec{v}_i)_{i = 1}^n
    \]
    where $f\colon \real^n \to \real$ is of smoothness $C^1$.
\end{definition}

Essentially this assigns every $n$-parallelotope in $\real^n$ to a scalar depending smoothly on the given point $P$.

\begin{lemma}[Top Forms are the Exterior Product of Basis Forms] \label{exterior_prod_theorem}
    $\omega$ is a top form iff
    \[
        \omega = f(P) \bigwedge_{i = 1}^n dx_i
    \]
    where $f\colon \real^n \to \real$ is of smoothness $C^1$ and $[dx_i]_{i = 1}^n$ are basis $1$-forms for $\real^n$.
\end{lemma}

Essentially \cref{exterior_prod_theorem} is saying that $\det = \bigwedge_{i = 1}^n dx_i$.

\begin{example}
    $\omega = xyz^2 dx \wedge dy \wedge dz$ is a $3$-form on $\real^3$. Notice that $\omega_P(\hat{i}, \hat{j}, \hat{k}) = f(P)$.
\end{example}

\subsection{Integrating Top Dimensional Forms}
To integrate an $n$-form we must integrate over a $n$-dimensional subset of $\real^n$.

\begin{definition}[$n$-Cell]
    An \textit{$n$-cell} is a solid, compact, $n$-dimensional, subset of $\real^n$.
\end{definition}

\begin{definition}[Parameterization of $n$-Cell]
    $c\colon \bigtimes_{i = 1}^n [a_i, b_i] \to \real^n$ is a \textit{parameterization} of an $n$-cell $R$ iff $c$ is of smoothness $C^1$ and $\range c = R$.
\end{definition}

\begin{example}
    $c\colon [0, 1] \times [0, 1] \to \real^2$ by $c(x, y) = (x, y)$ parameterizes the unit square in the obvious way.
\end{example}

\begin{example}
    $c\colon [0, 1] \times [0, 2\pi] \to \real^2$ by $c(r, \theta) = (r\cos\theta, r\sin\theta)$ parameterizes the unit disk using polar coordinates.
\end{example}

\begin{definition}[Integral of Top Form]
    If $\omega$ is a top form of $\real^n$, then the \textit{integral} of $\omega$ over an $n$-cell $R$ is given by
    \[
        \int_R \omega = \lim_{[k_i]_{i = 1}^n \to \infty} \left[ \sum_{j_i = 1}^{k_i} \right]_{i = 1}^n \omega_{P_{[j_i]_{i = 1}^n}}(\vec{v}_{[j_i]_{i = 1}^n})
    \]
    % where $[P_i]_{i = 0}^k \in C$ such that $P_0 = A$ and $P_k = B$ and $[\vec{v}_i = \overrightarrow{P_{i - 1}P_i}]_{i = 1}^k$.
\end{definition}

\begin{definition}[Jacobian]
    If $f\colon \real^n \to \real^m$, then the \textit{Jacobian} $Jf\colon \real^n \to \real^{m, n}$ is defined by
    \[
        Jf = [ D_j f ]_{j = 1}^n = ( D_j f_i )_{i = 1, j = 1}^{m, n}
    \]
    where $f = (f_i)_{i = 1}^m$.
\end{definition}

\begin{theorem}[Fubini's Theorem]
    If $\omega = f(P)\bigwedge_{i = 1}^n dx_i$ is a top form of $\real^n$ and $c\colon \bigtimes_{i = 1}^n [a_i, b_i] \to \real^n$ is a parameterization of $R$, then
    \[
        \int_R \omega = \left[ \int_{a_i}^{b_i} \right]_{i = 1}^n f(c(t_i)_{i = 1}^n) \det((Jc)(t_i)_{i = 1}^n) [dt_i]_{i = 1}^n
    \]
\end{theorem}

\begin{theorem}
    If $\omega$ is a top form of $\real^n$ and $c\colon \bigtimes_{i = 1}^n [a_i, b_i] \to \real^n$ is a parameterization of $R$, then
    \[
        \int_R \omega = \left[ \int_{a_i}^{b_i} \right]_{i = 1}^n \omega_{c(t_i)_{i = 1}^n}((D_i c)_{i = 1}^n(t_i)_{i = 1}^n) [dt_i]_{i = 1}^n
    \]
\end{theorem}

\begin{exercise}
    If $\omega = x^2 y dx \wedge dy$ and $c\colon [0, 1] \times [0, 1] \to \real^2$ as $c(s, t) = (s, t)$ is a parameterization of $R$, evaluate $\int_R \omega$.
\end{exercise}
\begin{solution}
    \begin{align*}
        \int_R \omega
        &= \int_0^1 \int_0^1 \omega_{c(s, t)}\left( \left(\frac{\del c}{\del s}, \frac{\del c}{\del t}\right)(s, t) \right)dtds
        = \int_0^1 \int_0^1 \omega_{(s, t)}((1, 0), (0, 1))dtds \\
        &= \int_0^1 \int_0^1 s^2 t \det((1, 0), (0, 1)) dtds
        = \int_0^1 \int_0^1 s^2 t dtds
        = \frac{1}{2} \int_0^1 s^2 t^2\Big|_{t = 0}^{t = 1} ds \\
        &= \frac{1}{2} \int_0^1 s^2 ds
        = \frac{1}{6} s^3\Big|_{s = 0}^{s = 1}
        = \frac{1}{6}
    \end{align*}
\end{solution}

\begin{exercise}
    If $\omega = (x^2 + y^2)dx\wedge dy$ and $c\colon [0, 1] \times [0, 2\pi] \to \real^2$ as $c(r, \theta) = (r\cos \theta, r\sin \theta)$ is a parameterization of $R$, evaluate $\int_R \omega$.
\end{exercise}
\begin{solution}
    {\allowdisplaybreaks
    \begin{align*}
        \int_R \omega
        &= \int_0^1 \int_0^{2\pi} \omega_{c(r, \theta)}\left( \frac{\del c}{\del r}, \frac{\del c}{\del \theta} \right)(r, \theta) drd\theta \\
        &= \int_0^1 \int_0^{2\pi} \omega_{c(r, \theta)}( (\cos\theta, \sin\theta), (-r\sin\theta, r\cos\theta) )drd\theta \\
        &= \int_0^1 \int_0^{2\pi} (r^2\cos^2\theta + r^2\sin^2\theta)\begin{vmatrix}
            \cos \theta & -r\sin\theta \\
            \sin \theta & r\cos\theta
        \end{vmatrix}drd\theta \\
        &= \int_0^1 \int_0^{2\pi} (r^2\cos^2\theta + r^2\sin^2\theta)(r\cos^2\theta + r\sin^2\theta)drd\theta \\
        &= \int_0^1 \int_0^{2\pi} r^3 drd\theta
        = 2\pi\int_0^1 r^3 dr
        = \frac{\pi}{2} r^4\Big|_{r = 0}^{r = 1}
        = \frac{\pi}{2}
    \end{align*}
    }
\end{solution}

\begin{theorem}[Chain Rule]
    If $f\colon \real^n \to \real^m$ and $g\colon \real^m \to \real^p$ are $C^1$ smooth then
    \[
        J(g \circ f)(P) = \Big((Jg)(f(P)) \Big) \Big( (Jf) (P) \Big)
    \]
\end{theorem}

% TODO: Reparameterization

\end{document}