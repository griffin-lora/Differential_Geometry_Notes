\documentclass[notes]{subfiles}
\externaldocument{determinants}

\begin{document}

\setcounter{section}{6}
\section{Differential k-forms}
\subsection{Definition of k-forms}

\begin{definition}[Differential $k$-Form]
    $\alpha$ is a \textit{$k$-form} \textit{on} $\real^n$ iff $\alpha$ is a map from every $k$-parallelotope in $\real^n$ to a scalar in $\real$ such that $\alpha$ is
    \begin{enumerate}[label = (\arabic*)]
        \item scaling with respect to the scale of any spanning vector
        \[
            \alpha_P(\ldots, t\vec{v}, \ldots) = t\alpha_P(\ldots, \vec{v}, \ldots)
        \]
        \item additive for each spanning vector
        \[
            \alpha_P(\ldots, \vec{u} + \vec{v}, \ldots) = \alpha_P(\ldots, \vec{u}, \ldots) + \alpha_P(\ldots, \vec{v}, \ldots)
        \]
        \item alternating with respect to swapping any two spanning vectors
        \[
            \alpha_P(\ldots, \vec{u}, \ldots, \vec{v}, \ldots) = -\alpha_P(\ldots, \vec{v}, \ldots, \vec{u}, \ldots)
        \]
        \item $C^1$ smooth with respect to $P$
    \end{enumerate}
\end{definition}

\begin{exercise} \label{basic_2_form_on_r3}
    Show that $\alpha$ such that
    \[
        \alpha_P(\vec{u}, \vec{v}) = \begin{vmatrix}
            a_1 & b_1 \\
            a_2 & b_2
        \end{vmatrix}
    \]
    is a $2$-form on $\real^3$
\end{exercise}
\begin{proof}
    For each of the $k$-form axioms we have
    \begin{enumerate}[label = (\arabic*)]
        \item \begin{align*}
            \alpha_P(t\vec{u}, \vec{v})
            &= \begin{vmatrix}
                ta_1 & b_1 \\
                ta_2 & b_2
            \end{vmatrix}
            = ta_1b_2 - tb_1a_2
            = t\begin{vmatrix}
                a_1 & b_1 \\
                a_2 & b_2
            \end{vmatrix}
            = \alpha_P(\vec{u}, \vec{v})
        \end{align*}
        and wlog for $\alpha_P(\vec{u}, t\vec{v}) = t\alpha_P(\vec{u}, \vec{v})$.
        \item \begin{align*}
            \alpha_P(\vec{u} + \vec{v}, \vec{w})
            &= \begin{vmatrix}
                a_1 + b_1 & c_1 \\
                a_2 + b_2 & c_2
            \end{vmatrix}
            = c_2a_1 + c_2b_1 - c_1a_2 - c_1b_2 \\
            &= (a_1c_2 - c_1a_2) + (b_1c_2 - c_1b_2)
            = \begin{vmatrix}
                a_1 & c_1 \\
                a_2 & c_2
            \end{vmatrix}
            +
            \begin{vmatrix}
                b_1 & c_1 \\
                b_2 & c_2
            \end{vmatrix} \\
            &= \alpha_P(\vec{u}, \vec{w}) + \alpha_P(\vec{v}, \vec{w})
        \end{align*}
        and wlog for $\alpha_P(\vec{u}, \vec{v} + \vec{w}) = \alpha_P(\vec{u}, \vec{v}) + \alpha_P(\vec{u}, \vec{w})$
        \item \begin{align*}
            \alpha_P(\vec{u}, \vec{v})
            &= \begin{vmatrix}
                a_1 & b_1 \\
                a_2 & b_2
            \end{vmatrix}
            = a_1b_2 - b_1a_2
            = -(b_1a_2 - a_1b_2) \\
            &= -\begin{vmatrix}
                b_1 & a_1 \\
                b_2 & a_2
            \end{vmatrix}
            = -\alpha_P(\vec{v}, \vec{u})
        \end{align*}
        \item Since $\alpha$ does not vary with choice of $P$, then $\alpha$ is $C^1$ smooth with respect to $P$.
    \end{enumerate}
    % \begin{enumerate}[label = (\arabic*)]
    %     \item Since the determinant scales with respect to the scale of any column vector, then $\alpha$ scales with respect to any of its spanning vectors.
    %     \item By \cref{det_additive} we know that the determinant is additive with respect to any of its column vectors, therefore $\alpha$ is additive with respect to any of its column vectors.
    %     \item Since the determinant is alternating with respect to any of its column vectors, then $\alpha$ is alternating with respect to any of its spanning vectors.
    %     \item Since $\alpha$ does not vary with choice of $P$, then $\alpha$ is $C^1$ smooth with respect to $P$.
    % \end{enumerate}
\end{proof}

Notice that axioms $1$ to $3$ are all satisfied by the determinant, therefore if we take the determinant of spanning vectors, as long as axiom $4$ is also satisfied, then we will end up with a $k$-form over $\real^n$.

Also notice that in \cref{basic_2_form_on_r3} $\alpha_P(\vec{u}, \vec{v})$ will give us the oriented area of the parallelogram spanned by $\vec{u}$ and $\vec{v}$ when projected down to the $xy$-plane.



\end{document}