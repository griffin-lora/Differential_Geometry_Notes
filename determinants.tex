\documentclass[notes]{subfiles}

\begin{document}

\setcounter{section}{4}
\section{Determinants}
\subsection{Definition and Basic Properties}
\begin{definition}[Matrix]
    An $m$ by $n$ \textit{matrix} over a field $K$ is
    \[
        K^{m, n} = \bigtimes_{i = 1}^n K^m
    \]
\end{definition}

\begin{definition}[Determinant]
    A \textit{determinant} of \textit{dimension} $n$ for a \textit{basis} $\{ \vec{e}_i \}_{i = 1}^n$ of $\real^n$ is a function
    \[
        D\colon \real^{n, n} \to \real
    \]
    such that it is
    \begin{enumerate}[label = (\arabic*)]
        \item invariant (up to said scalar) under the scaling of any of its column vectors
        \[
            D(\ldots, t\vec{v}, \ldots) = tD(\ldots, \vec{v}, \ldots)
        \]
        \item alternating with respect to swapping of column vectors
        \[
            D(\ldots, \vec{u}, \ldots, \vec{v}, \ldots) = -D(\ldots, \vec{v}, \ldots, \vec{u}, \ldots)
        \]
        \item invariant under adding linear combinations of column vectors together
        \[
            D(\ldots, \vec{u}, \ldots, \vec{v}, \ldots) = D(\ldots, \vec{u} + t\vec{v}, \ldots, \vec{v}, \ldots)
        \]
        \item one on the identity matrix of $E$ of $\real^n$
        \[
            D(E) = 1
        \]
    \end{enumerate}
\end{definition}

\begin{lemma}[Determinant is Unique]
    If there exists an $n$-dimensional determinant $D$, then $D$ is unique up to of dimension.
\end{lemma}

Therefore we denote the determinant $D$ as $D = \det$.

\begin{theorem}[Determinant is Additive with Respect to Each Column Vector] \label{det_additive}
    \[
        \det(\ldots, \vec{u} + \vec{v}, \ldots) = \det(\ldots, \vec{u}, \ldots) + \det(\ldots, \vec{v}, \ldots)
    \]
\end{theorem}

\begin{lemma}[Determinant is Multiplicative]
    $\det(AB) = \det(A)\det(B)$
\end{lemma}

\begin{lemma}
    $\det(\ldots, 0, \ldots) = 0$
\end{lemma}
\begin{proof}
    \begin{align*}
        \det(\ldots, 0, \ldots)
        &= \det(\ldots, 0\vec{v}, \ldots)
        = 0\det(\ldots, \vec{v}, \ldots)
        = 0
    \end{align*}
\end{proof}

\subsection{Relationship to Volume}
First, we look at parallelograms in $\real^2$.

\begin{definition}[Parallelogram]
    If $\vec{v}, \vec{w} \in \real^2$ then
    \[
        P(\vec{v}, \vec{w}) = \{ s\vec{v} + t\vec{w} : 0 \leq s \leq 1, 0 \leq t \leq 1 \}
    \]
    is the \textit{parallelogram} that is \textit{spanned} by $\vec{v}$ and $\vec{w}$.
\end{definition}

If we change the order of $\vec{v}$ and $\vec{w}$ then we get the corresponding parallelogram with opposite orientation.

\begin{definition}[Oriented Area]
    $A\colon \real^2 \times \real^2 \to \real^2$ is the \textit{oriented area} function for parallelograms iff $A(\vec{v}, \vec{w})$ gives the area of $P(\vec{v}, \vec{w})$ if $P(\vec{v}, \vec{w})$ has positive orientation and $A(\vec{v}, \vec{w})$ gives the negative area of $P(\vec{v}, \vec{w})$ if $P(\vec{v}, \vec{w})$ has negative orientation.
\end{definition}

\begin{lemma}
    The oriented area function is equal to the determinant.
\end{lemma}
\begin{proof}
    For axiom 1, scaling any side of a parallelogram by $t$ will scale the area by $t$ so $A(t\vec{v}, \vec{w}) = A(\vec{v}, t\vec{w}) = tA(\vec{v}, \vec{w})$.

    For axiom 2, $A(\vec{v}, \vec{w}) = -A(\vec{w}, \vec{v})$ since swapping $\vec{v}$ and $\vec{w}$ switches the orientation.

    For axiom 3, $P(\vec{v} + t\vec{w}, \vec{w})$ and $P(\vec{v}, \vec{w} + t\vec{v})$ are sheared parallelograms, since area is invariant under shearing then $A(\vec{v} + t\vec{w}, \vec{w}) = A(\vec{v}, \vec{w} + t\vec{v}) = A(\vec{v},vec{w})$.

    For axiom 4, since the unit basis vectors all have length $1$, then the area must be $1$, therefore $A((1, 0), (0, 1)) = 1$.

    Therefore $A = \det$.
\end{proof}

\begin{theorem}
    The oriented volume of an $n$-parallelotope spanned by $[\vec{v}_i]_{i = 1}^n$ is $\det(\vec{v}_i)_{i = 1}^n$.
\end{theorem}

\end{document}